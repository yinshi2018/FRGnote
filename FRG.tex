\documentclass[UTF8]{article}
\usepackage{appendix}
\usepackage{amsmath}
\usepackage{dsfont}
\usepackage{amsmath}
\usepackage{amssymb}
\usepackage{amsthm}
\usepackage{graphicx}
\usepackage{multirow}
\usepackage{multicol,ifthen,booktabs,amsmath,amsfonts,bm,mathrsfs,amssymb}
\usepackage{times,mathptmx}
\usepackage{geometry}
\usepackage{slashed}
\renewcommand\baselinestretch{1.5}\protect
\abovedisplayshortskip 3 pt
\belowdisplayshortskip 3 pt
\geometry{left=2cm,right=2cm,top=3cm,bottom=3cm}
\begin{document}
%\author{Shi Yin}
%\affiliation{School of Physics, Dalian University of Technology, Dalian, 116024,P.R. China}
\title{FRG note}
\maketitle
\newpage
\tableofcontents
\newpage
%%%%%%%%%%%%%%%%%%%%%%%%%%%%%%%%
\section{Wetterich Equation}
Consider the generating function which is cutoff at infrared:
\begin{equation}\label{eq0001}
Z_k[J]=\exp(W_k[J])=\int [d\phi]\exp\{-S[\phi]-\Delta S_k[\phi]]+J^a\phi_a\}
\end{equation}
The $\phi$ stands for every kind of field here. Index $a$ stands for every degree of freedom, including different fields, different components of a field. Such as the space-time coordinates or momentum index.
$S[\phi]$ is the classical action, $J^a$ is the source of $\phi_a$. $\Delta S_k[\phi]$ is the infrared cutoff,
 its function is to cutoff the quantum fluctuation at $p^2\leq k^2$ and keep the fluctuation at $p^2>k^2$ invariant.
  We usually choose the form of quadratic term (mass term) to achieve the infrared cutoff. 
\begin{equation}\label{eq0002}
\Delta S_k[\phi]=\frac{1}{2}\phi_aR_{k}^{ab}\phi_b
\end{equation}
with $R_{k}^{ab}=R_{k}^{ba}$ ($a,b$ is the index of the boson), $R_{k}^{ab}=-R_{k}^{ba}$ ($a,b$ is the index of the fermion). Here we give a example of a simple scalar field. In the coordinate space:
\begin{equation}
\Delta S_k[\phi]=
\cfrac{1}{2}\int d^4xd^4y\varphi(x)R_k(x,y)\varphi(y)
\end{equation}
So 
\begin{equation}
\begin{split}
R_k(x,y)= & \int\cfrac{d^4p}{(2\pi)^4}R_k(p,-p)e^{ip(x-y)}\\
= & \int\cfrac{d^4p}{(2\pi)^4} R_k(p)e^{ip(x-y)}
\end{split}
\end{equation}
\begin{equation}
\varphi(x)=\int\cfrac{d^4p}{(2\pi)^4}e^{ipx}\varphi(p)
\end{equation}
\begin{equation}
\begin{split}
\Delta S_k[\varphi]=&\cfrac{1}{2}\int d^4xd^4y\varphi(x)R_k(x,y)\varphi(y)\\
=&\cfrac{1}{2}\int d^4xd^4y
\int\cfrac{d^4p_1}{(2\pi)^4}e^{ip_1x}\varphi(p_1)
\int\cfrac{d^4p}{(2\pi)^4}R_k(p)e^{ip(x-y)}
\int\cfrac{d^4p_2}{(2\pi)^4}e^{ip_2y}\varphi(p_2)\\
=&\cfrac{1}{2}\int\cfrac{d^4p}{(2\pi)^4}\varphi(-p)R_k(p)\varphi(p)\\
=&\cfrac{1}{2}\int\cfrac{d^4q}{(2\pi)^4}\varphi(-q)R_k(q)\varphi(q)
\end{split}
\end{equation}
For the determined $q$ , the regulator $R_k(q)$ satisfies the following properties
\begin{equation}
R_{k\rightarrow\infty}(q)\rightarrow\infty
\quad
R_{k\rightarrow 0}(q)\rightarrow 0
\end{equation}
To suppress the fluctuation at $q^2<k^2$ and keep the fluctuation at high momentum unchanging, we should choose
\begin{equation}
R_k(q)|_{q^2<k^2}\sim k^2,
\quad
R_k(q)|_{q^2>k^2}\sim 0
\end{equation}
for example
\begin{equation}
R_k(q)\sim\cfrac{q^2}{e^{\frac{q^2}{k^2}}-1}
\end{equation}
of course, we can choose other forms of the regulator. From \eqref{eq0001} we can get
\begin{equation}
\begin{split}
\cfrac{\delta W_k[J]}{\delta J^a}=&\cfrac{1}{Z_k}
\cfrac{\delta Z_k[J]}{\delta J^a}\\
=&\cfrac{1}{Z_k}\int [d\phi]\phi_a\exp\{-S[\phi]
-\Delta S_k[\phi]+J^a\phi_a\}\\
=&\langle\phi_a\rangle
\end{split}
\end{equation}
In the below discussing, we replace $\langle\phi_a\rangle$ with $\phi_a$ and
\begin{equation}
\cfrac{\delta^2W_k[J]}{\delta J^b\delta J^a}
=\langle \phi_b\phi_a\rangle_c\equiv G_{ba}^k
\end{equation}
index $c$ stands for the connected diagram, $G$ is the propagator that depends on the scale $k$.
 Now we do the Legendre transformation on the generated functional of connected diagram,
  then we can obtain the generated functional of one-particle irreducible diagram, that is the effective action
\begin{equation}\label{eq0003}
\Gamma_k[\phi]=-W_k[J]+J^a\phi_a-\Delta S_k[\phi]
\end{equation}
Beware here $\phi_a \equiv \langle \phi_k \rangle$. In order to consider the boson and fermion together,
 we introduce the following symbols
\begin{equation}
J^a\phi_a=r^{a}_{b}\phi_aJ^b
\end{equation}
with
\begin{equation}
r^{a}_{b}=(-1)^{ab}\delta^{a}_{b}
\end{equation}
\begin{equation}
(-1)^{ab}\equiv
\begin{cases}
-1,\quad for \quad a,b \quad Fermionic\\
1,\quad for \quad a,b \quad Bosonic\\
\end{cases}
\end{equation}
So from \eqref{eq0003} we obtain
\begin{equation}
\cfrac{\delta(\Gamma_k[\phi]+\Delta S_k[\phi])}{\delta\phi_a}
=r_b^aJ^b
\end{equation}
Differentiate $J$ on both sides of the above formula
\begin{equation}
\cfrac{\delta^2(\Gamma_k[\phi]+\Delta S_k[\phi])}
{\delta J^b\delta\phi_a}
=r_b^a
\end{equation}
\begin{equation}
\begin{split}
l.h.s=& \cfrac{\delta^2(\Gamma_k[\phi]+\Delta S_k[\phi])}
{\delta J^b\delta\phi_a}
\\=& \cfrac{\delta\phi_c}{\delta J^b}\cfrac{\delta^2(\Gamma_k[\phi]+\Delta S_k[\phi])}
{\delta\phi_c\delta\phi_a}\\
=& G^{k}_{bc}\cfrac{\Gamma_k[\phi]+\Delta S_k[\phi]}{\delta\phi_c\delta\phi_a}\\
\end{split}
\end{equation}
We obtain
\begin{equation}
G^{k}_{bc}\cfrac{\Gamma_k[\phi]+\Delta S_k[\phi]}{\delta\phi_c\delta\phi_a}=r^{a}_{b}
\end{equation}
\begin{equation}
\cfrac{\Gamma_k[\phi]+\Delta S_k[\phi]}{\delta\phi_c\delta\phi_a}
=(\Gamma^{(2)}_{k}[\phi]+R_k)^{ca}
\end{equation}
then
\begin{equation}
G^{k}_{bc}(\Gamma^{(2)}_{k}[\phi]+R_k)^{ca}=r^{a}_{b}
G^{k}_{bc}=r^{a}_{b}(\Gamma^{(2)}_{k}[\phi]+R_k)^{-1}_{ac}
\end{equation}
Now we calculate $\partial_tW_k[J]=k\cfrac{\partial}{\partial k}W_k[J]$
\begin{equation}
\begin{split}
\partial_tW_k[J]=& \partial_tlnZ_k[J]\\
=& \cfrac{1}{Z_k}\partial_tZ_k[J]\\
=& \cfrac{1}{Z_k}\int[d\phi](-\partial_t\Delta S_k[\phi])e^{-S[\phi]-\Delta S_k[\phi]+J^a\phi_a}\\
=& -\cfrac{1}{Z_k}\int[d\phi]\frac{1}{2}\phi_a\partial_tR^{ab}_{k}\phi_be^{-S[\phi]-\Delta S_k[\phi]+J^a\phi_a}\\
=& -\frac{1}{2}\langle\phi_a\phi_b\rangle\partial_tR^{ab}_{k}\\
=& -\frac{1}{2}(\langle\phi_a\phi_b\rangle_c+\langle\phi_a\rangle\langle\phi_b\rangle)\partial_tR^{ab}_{k}
\end{split}
\end{equation}
we replace $\langle\phi_a\rangle$ with $\phi_a$ again
\begin{equation}
\langle\phi_a\phi_b\rangle_c=
\begin{cases}
\langle\phi_b\phi_a\rangle,\quad for \quad a,b \quad Bosonic\\
-\langle\phi_b\phi_a\rangle,\quad for \quad a,b \quad Fermionic\\
\end{cases}
\end{equation}
Then we obtain
\begin{equation}\label{eq0004}
\partial_tW_k[J]=-\frac{1}{2}STr{G_k(\partial_tR_k)}-\frac{1}{2}\phi_a\partial_tR^{ab}_{k}\phi_b
\end{equation}
$STr$ is super trace, including the trace of every field and dispersed, continuously degree of freedom. For the fermion a negative sign should be added.
Finally, from \eqref{eq0003} and \eqref{eq0004} we obtain
\begin{equation}
\begin{split}
\partial_t\Gamma_k[\phi]=& -\partial_tW_k[J]-\partial_t\Delta S_k[\phi]\\
=& \frac{1}{2}\phi_a\partial_tR^{ab}_{k}\phi_b-\frac{1}{2}\phi_a\partial_tR^{ab}_{k}\phi_b\\
=& \frac{1}{2}STr{G_k(\partial_tR_k)}
\end{split}
\end{equation}
This is the Wetterich equation.
%%%%%%%%%%%%%%%%%%%%%%%%%%%%%%%%%%%
\section{QCD}
In this section, we introduce the application of FRG in QCD. We start from the effective action that depends on the infrared cutoff scale $k$
\begin{equation}
\begin{split}
\Gamma_k=&\int d^4x\{ \frac{1}{4}F^{a}_{\mu\nu}F^{a}_{\mu\nu}+Z_{c,k}(\partial_\mu\overline c^a)D^{ab}_{\mu}c^b+\frac{1}{2\xi}(\partial_\mu A^{a}_{\mu})\\
+&Z_{q,k}\overline q(\gamma_\mu D_\mu)q-\lambda_{q,k}[(\overline qT^0q)^2-(\overline q\gamma_5T^aq)^2]\\
+&h_k[\overline q(i\gamma_5T^a\pi^a+T^0\sigma)q]+\frac{1}{2}Z_{\phi,k}(\partial_\mu\phi)^2 \}
\end{split}
\end{equation}
In the effective action 
\begin{equation}
\begin{split}
&\phi=(\sigma,\pi^a) , \rho=\frac{1}{2}\phi^2\\
&D_\mu=\partial_\mu-iZ^{\frac{1}{2}}_{A,k}g_kA^{a}_{\mu}t^a\\
&D^{ab}_{\mu}=\partial_\mu\delta^{ab}+Z^{\frac{1}{2}}_{A,k}g_kf^{acb}A^{c}_{\mu}
\end{split}
\end{equation}
In the definition above $T^a (a=1,2,...,N_f^2-1)$ is the generator of flavor space with $T^0=\cfrac{1}{\sqrt{2N_f}} \mathds{1}$; $t^a(a=1,2,...,N_c^2-1)$ is the generator of color space.
\begin{equation}
\begin{split}
&F_{\mu\nu}=\frac{i}{g}[D_\mu,D_\nu]=F^{a}_{\mu\nu}t^a\\
&F^{a}_{\mu\nu}=Z^{\frac{1}{2}}_{A,k} (\partial_\mu A^{a}_{\nu}-\partial_\nu A^{a}_{\mu}+Z^{\frac{1}{2}}_{A,k} g_k f^{abc}A^b_\mu A^c_\nu)
\end{split}
\end{equation}
We rewrite the Wetterich equation as below
\begin{equation}\label{weq2}
\partial_t\Gamma_k[\Phi]=\frac{1}{2}STr{\widetilde{\partial_t}ln(\Gamma^{(2)}_{k}+R_k)}
\end{equation}
the tilde on the $\partial_t$ stands for the derivation only works on the regulator $R_k$.
\begin{equation}
(\Gamma^{(2)}_{k})_{ab}=\cfrac{\overrightarrow{\delta}}{\delta\Phi^{T}_{a}}\Gamma_k\cfrac{\overleftarrow{\delta}}{\delta\Phi_{b}}
\end{equation}
the definition of $\Phi$ is
\begin{equation}
\begin{split}
\Phi=&
\begin{pmatrix}
 A(q)\\\sigma(q)\\\pi(q)\\q(q)\\\overline{q}^T(q)
\end{pmatrix}\\
\Phi^T=&
\begin{pmatrix}
A^T(-q),\sigma(-q),\pi(-q),q^T(q),\overline{q}(q)
\end{pmatrix}\\
\end{split}
\end{equation}
the fluctuation matrix in the \eqref{weq2} can be rewrite as
\begin{equation}
\Gamma^{(2)}_{k}+R_k=P+F
\end{equation}
The matrix $P$ contains the propagators and regulators ; matrix $F$ contains the dependence of 
all kind of fields. So we can expanse the \eqref{weq2} with the number of the fields
\begin{equation}
\begin{split}
\partial_t\Gamma_k[\Phi]=&\frac{1}{2}STr\widetilde{\partial_t}ln(P+F)\\
=&\frac{1}{2}STr\widetilde{\partial_t}ln[P(1+\frac{1}{P}F)]\\
\sim&\frac{1}{2}STr\widetilde{\partial_t}ln(1+\frac{1}{P}F)\\
=&\frac{1}{2}STr\{\widetilde{\partial_t}[\frac{1}{P}F-\frac{1}{2}(\frac{1}{P}F)^2+\frac{1}{3}
(\frac{1}{P}F)^3-\frac{1}{4}(\frac{1}{P}F)^4+\cdots]\}\\
=&\frac{1}{2}STr\widetilde{\partial_t}(\frac{1}{P}F)-\frac{1}{4}STr\widetilde{\partial_t}
(\frac{1}{P}F)^2+\frac{1}{6}STr\widetilde{\partial_t}(\frac{1}{P}F)^3-\frac{1}{8}STr\widetilde
{\partial_t}(\frac{1}{P}F)^4+\cdots
\end{split}
\end{equation}
then we take the contribution of the lowest order in
\begin{equation}
\begin{split}
\partial_t\Gamma_k[\Phi]=&\frac{1}{2}STr\widetilde{\partial_t}ln(P+F)\\
=&\frac{1}{2}STr\widetilde{\partial_t}ln[P(1+\frac{1}{P}F)]\\
=&\frac{1}{2}STr\widetilde{\partial_t}[lnP+ln(1+\frac{1}{P}F)]\\
=&\frac{1}{2}STr\widetilde{\partial_t}lnP+\frac{1}{2}STr\widetilde{\partial_t}ln(1+\frac{1}{P}F)\\
=&\frac{1}{2}STr\widetilde{\partial_t}lnP+\frac{1}{2}STr\widetilde{\partial_t}(\frac{1}{P}F)
-\frac{1}{4}STr\widetilde{\partial_t}(\frac{1}{P}F)^2
+\frac{1}{6}STr\widetilde{\partial_t}(\frac{1}{P}F)^3
-\frac{1}{8}STr\widetilde{\partial_t}(\frac{1}{P}F)^4+\cdots
\end{split}
\end{equation}
%%%%%%%%%%%%%%%%%%%%%%%%%%%%%%%%%%%
%介子场
\subsection{Meson field propagator}
The meson propagator part of the effective potential and its Fourier transform
\begin{equation}
\begin{split}
&\int d^4x \frac{1}{2}Z_{\phi,k}(\partial_\mu\phi)^2\\
=&\int\frac{d^4q}{(2\pi)^4}\frac{1}{2}Z_{\phi,k}\phi(-q)q^2\phi(q)
\end{split}
\end{equation}
then consider the contribution of the meson mass
\begin{equation}
\begin{split}
\Gamma^{\sigma\sigma}_{k}(q',q)\equiv&\frac{\delta^2\Gamma_k}
{\delta\sigma(q')\delta\sigma(q)}\\
=&(Z_{\phi,k}q^2+m^{2}_{\sigma})(2\pi)^4\delta^4(q+q')
\end{split}
\end{equation}
The corresponding cutoff function is
\begin{equation}
\begin{split}
R^{\sigma\sigma}_{k}(q',q)=Z_{\phi,k}\vec{q}^2r_B(\frac{\vec{q}^2}{k^2})(2\pi)^4\delta^4(q+q')
\end{split}
\end{equation}
here we use the 3d regulator which is convenient to the calculation of the finite temperature.
The $r_B(x)=(\frac{1}{x}-1)\theta(1-x)$ is an optimized regulator.\\
Then we treat the Pi meson in the same way
\begin{equation}
\begin{split}
\Gamma^{\pi\pi}_{k,ij}(q',q)\equiv&\frac{\delta^2\Gamma_k}
{\delta\pi_i(q')\delta\pi_j(q)}\\
=&\delta_{ij}(Z_{\phi,k}q^2+m^{2}_{\pi})(2\pi)^4\delta^4(q+q')
\end{split}
\end{equation}
The cutoff function is
\begin{equation}
R^{\pi\pi}_{k,ij}=\delta_{ij}R^{\sigma\sigma}_{k}
\end{equation}
In the calculation of finite temperature we have $q_0=2\pi nT$ in which the $n$ is integer.
%%%%%%%%%%%%%%%%%%%%%%%%%%%%%%%%%%%
%夸克场
\subsection{Quark field propagator}
The quark propagator part and its Fourier transform
\begin{equation}
\begin{split}
&\int d^4x Z_{q,k}\overline{q}(x)(\gamma_\mu\partial_\mu+\frac{m_f}{Z_{q,k}})q(x)\\
=&\int\frac{d^4q}{(2\pi)^4}Z_{q,k}\overline{q}(q)(i\slashed{q}+\frac{m_f}{Z_{q,k}})q(q)
\end{split}
\end{equation}
The Fourier transform of the quark and anti-quark fields 
\begin{equation}
\begin{split}
&q(x)=\int \frac{d^4q}{(2\pi)^4}e^{iqx}q(q) \\
&\overline{q} (x)=\int\frac{d^4q}{(2\pi)^4}e^{-iqx}\overline{q}(q)
\end{split}
\end{equation}
The derivation of the effective action amount of the quark fields is
\begin{equation}
\begin{split}
\Gamma^{\overline{q}q}_{k,ij}(q',q)&\equiv\frac{\overrightarrow{\delta}}{\delta q_i(q')}\Gamma_k\frac{\overleftarrow{\delta}}{\delta\overline{q}_j(q)}\\
&=(Z_{q,k}i\slashed{q}_{ij}+m_f\delta_{ij})(2\pi)^4\delta^4(q'-q)\\
&=(Z_{q,k}iq_\mu(\gamma_\mu)_{ij}+m_f\delta_ij)(2\pi)^4\delta^4(q'-q)
\end{split}
\end{equation}
\begin{equation}
\begin{split}
\Gamma^{q\overline{q}}_{k,ij}&\equiv\frac{\overrightarrow{\delta}}{\delta q_i(q')}\Gamma_k\frac{\overleftarrow{\delta}}{\delta\overline{q}_j(q)}\\
&=-(Z_{q,k}iq_\mu(\gamma_\mu)_{ji}+m_f\delta_{ji})(2\pi)^4\delta^4(q'-q)\\
\end{split}
\end{equation}
The corresponding regulator function is 
\begin{equation}
R^{\overline{q}q}_{k,ij}(q',q)=Z_{q,k}i\vec{q}\cdot\vec{\gamma}_{ij}r_F(\frac{\vec{q}^2}{k^2})(2\pi)^4\delta^4(q'-q)
\end{equation}
with
\begin{equation}
r_F(x)=(\frac{1}{\sqrt{x}}-1)\theta(1-x)
\end{equation}
%%%%%%%%%%%%%%%%%%%%%%%%%%%%%%%%%%%%%%
%胶子场
\subsection{Gluon field propagator}
The differential form of the effective action is 
\begin{equation}
\begin{split}
(\Gamma^{AA}_{k})^{ab}_{\mu\nu}(q',q)&\equiv\frac{\delta^2\Gamma_k}{\delta A^{a}_{\mu}(q') \delta A^{b}_{\nu}(q)}\\
&=[Z_{A,k}q^2(\delta_{\mu\nu}-\frac{q_\mu q_\nu}{q^2})+\frac{q^2}{\xi}(\frac{q_\mu q_\nu}{q^2})]\delta^{ab}(2\pi)^4\delta^4(q'+q)
\end{split}
\end{equation}
The regulator function is
\begin{equation}
\begin{split}
(R^{AA}_{k})^{ab}_{\mu\nu}(q',q)=[Z_{A,k}\vec{q}^2r_B(\frac{\vec{q}^2}{k^2})(\delta_{\mu\nu}-\frac{q_\mu q_\nu}{q^2})+\frac{\vec{q}^2}{\xi}
r_B(\frac{\vec{q}}{k^2})(\frac{q_\mu q_\nu}{q^2})]\delta^{ab}(2\pi)^4\delta^4(q'+q)
\end{split}
\end{equation}
The propagator of the gluon is
\begin{equation}
\begin{split}
(G^{AA}_{k})^{ab}_{\mu\nu}(q',q)=[\frac{1}{Z_{A,k}(q_0^2+\vec{q}^2(1+r_B))}(\delta_{\mu\nu}-\frac{q_\mu q_\nu}{q^2})+\frac{\xi}{q_0^2+\vec{q}
^2(1+r_B)}\frac{q_\mu q_\nu}{q^2}]\delta^{ab}(2\pi)^4\delta^4(q'+q)
\end{split}
\end{equation}
In the following calculation we adopt the Landau gauge $\xi=0$, then we can obtain the matrix $P$ \\
$P=$
\begin{equation}
\begin{pmatrix}
\begin{smallmatrix}
Z_{A,k}(q_0^2+\vec{q}^2(1+r_B))(\delta_{\mu \nu}-\frac{q_\mu q_\nu}{q^2})\delta_{ab} & 0 & 0 & 0 & 0 \\
0 & Z_{\phi,k}(q_0^2+\vec{q}^2(1+r_B))+m_{\sigma}^{2} & 0 & 0 & 0\\
0 & 0 & [Z_{\phi,k}(q_0^2+\vec{q}^2(1+r_B))+m_{\pi}^{2}]\delta_{ij} & 0 & 0\\
0 & 0 & 0 & 0 & -(Z_{q,k}i(q_0r_0+\vec{q}\cdot\vec{r}(1+r_F))+m_f)^T\\
0 & 0 & 0 & Z_{q,k}i(q_0r_0+\vec{q}\cdot\vec{r}(1+r_F))+m_f & 0
\end{smallmatrix}
\end{pmatrix}
\end{equation}
And then we can obtain the propagator matrix
\begin{equation}
\frac{1}{P}=
\begin{pmatrix}
(G^{AA}_{k})^{ab}_{\mu\nu} & 0 & 0 & 0 & 0 \\
0 & G^{\sigma}_{k} & 0 & 0 & 0\\
0 & 0 & (G^{\pi}_{k})_{ij} & 0 & 0\\
0 & 0 & 0 & 0 & G^{q\overline{q}}_{k}\\
0 & 0 & 0 & G^{\overline{q}q}_{k} & 0
\end{pmatrix}
\end{equation}
the definition of the matrix elements are
\begin{equation}
\begin{split}
(G^{AA}_{k})^{ab}_{\mu\nu}&=\frac{1}{Z_{A,k}(q_0^2+\vec{q}^2(1+r_B))}(\delta_{\mu\nu}-\frac{q_\mu q_\nu}{q^2})\delta_{ab}\\
G^{\sigma}_{k}&=\frac{1}{Z_{\phi,k}(q_0^2+\vec{q}^2(1+r_B))+m_{\sigma}^{2}}\\
(G^{\pi}_{k})_{ij}&=\frac{1}{Z_{\phi,k}(q_0^2+\vec{q}^2(1+r_B))+m_{\pi}^{2}}\delta_{ij}\\
G^{q\overline{q}}_{k}&=\frac{-Z_{q,k}i(q_0r_0+\vec{q}\cdot\vec{r}(1+r_F))+m_f}{Z_{q,k}^{2}(q_0^2+\vec{q}^2(1+r_F)^2)+m_f^2}\\
G^{\overline{q}q}_{k}&=-(G^{q\overline{q}}_{k})^T
\end{split}
\end{equation}
%%%%%%%%%%%%%%%%%%%%%%%%%%%%%%%%%%%%%
%场依赖部分F
%胶子顶点
\subsection{Gluon vertex}
First is the gauge part
\begin{equation}
\begin{split}
\Gamma_k&\sim\frac{1}{4}F^{a}_{\mu\nu}F^{a}_{\mu\nu}\\
&=\frac{1}{4}Z_{A,k}(\partial_\mu A^{a}_{\nu}-\partial_\nu A^{a}_{\mu}+Z_{A,k}^{\frac{1}{2}}g_kf^{abc}A^b_\mu A^c_\nu)(\partial_\mu A^{a}_{\nu}-\partial_\nu 
A^{a}_{\mu}+Z_{A,k}^{\frac{1}{2}}g_kf^{ab'c'}A^{b'}_\mu A^{c'}_\nu)\\
&\sim \frac{1}{2}Z^{\frac{3}{2}}_{A,k}(\partial_\mu A^{a}_{\nu}-\partial_\nu A^{a}_{\mu})g_kf^{abc}A^{b}_{\mu}A^{c}_{\nu}+\frac{1}{4}Z^{2}_{A,k}g^{2}_{k}
f^{abc}f^{ab'c'}A^{b}_{\mu}A^{c}_{\nu}A^{b'}_\mu A^{c'}_\nu
\end{split}
\end{equation}
here we let
\begin{equation}
\Gamma_1=(\partial_\mu A^{a}_{\nu}-\partial_\nu A^{a}_{\mu})f^{abc}A^{b}_{\mu}A^{c}_{\nu}
\end{equation}
then we can calculate the derivative
%\begin{equation}
%\begin{split}
%\frac{\delta\Gamma_1}{\delta A^{a}_{\mu}}&=2(\partial_\mu A^{b}_{\nu}-\partial_\nu A^{b}_{\mu})f^{bac}A^{c}_{\nu}\\
%&=2f^{abc}(\partial_\nu A^{b}_{\mu}-\partial_\mu A^{b}_{\nu})A^{c}_{\nu}\\
%&=2f^{abc}(\partial_\nu A^{b}_{\mu})A^{c}_{\nu}-2f^{abc}(\partial_\mu A^{b}_{\nu})A^{c}_{\nu}\\
%\end{split}
%\end{equation}
%\begin{equation}
%\begin{split}
%\frac{\delta^2\Gamma_1}{\delta A^{a}_{\mu}\delta A^{b}_{\nu}}&=2f^{ab'b}(\partial_\nu A^{b'}_{\mu})-2f^{abc}\delta_{\mu\nu}(\partial_\rho A^{c}_{\rho})-
%2f^{ab'b}(\partial_\mu A^{b'}_{\nu})+2f^{abc}(\partial_\mu A^{c}_{\nu})\\
%&=2f^{abc}(\partial_\mu A^{c}_{\nu}-\partial_\nu A^{c}_{\mu}+\partial_\mu A^{c}_{\nu}-\delta_{\mu\nu}\partial_\rho A^{c}_{\rho})
%\end{split}
%\end{equation}
\begin{equation}
\frac{\delta\Gamma_1}{\delta A^{a}_{\mu}}=-2(\partial_\nu)_af^{abc}A^{b}_{\mu}A^{c}_{\nu}+2f^{abc}(\partial_\nu A^{b}_{\mu}-\partial_\mu A^{b}_{\nu})
A^{c}_{\nu}
\end{equation}
\begin{equation}
\begin{split}
\frac{\delta^2\Gamma_1}{\delta A^{a}_{\mu}\delta A^{b}_{\nu}}&=-2(\partial_\rho)_af^{abc}\delta_{\mu\nu}A^{c}_{\rho}-2(\partial_\nu)_af^{acb}A^{c}_{\mu}
+2f^{abc}(\partial_\rho)_b\delta_{\mu\nu}A^{c}_{\rho}+2f^{acb}\partial_\nu A^{c}_{\mu}-2f^{abc}(\partial_\mu)_bA^{c}_{\nu}-2f^{acb}\partial_\mu A^{c}_{\nu}\\
&=2\delta_{\mu\nu}f^{abc}[(\partial_\rho)_b-(\partial_\rho)_a]A^{c}_{\rho}+2f^{abc}[(\partial_\nu)_aA^{c}_{\mu}-(\partial_\mu)_bA^{c}_{\nu}]+2f^{abc}
(\partial_\mu A^{c}_{\nu}-\partial_\nu A^{c}_{\mu})
\end{split}
\end{equation}
\begin{equation}
\begin{split}
\frac{\delta^3\Gamma_1}{\delta A^{a}_{\mu}\delta A^{b}_{\nu}\delta A^{c}_{\rho}}&=2\delta_{\mu\nu}f^{abc}[(\partial_\rho)_b-(\partial_\rho)_a]+2\delta_{\mu\nu}f^{abc}[(\partial_\nu)_a \delta_{\mu\rho}-(\partial_\mu)_b \delta_{\nu\rho}]+2f^{abc}[(\partial_\mu)_c \delta_{\rho\nu}-(\partial_\nu)_c \delta_{\mu\rho}]\\
&=2f^{abc}\bigg\{\delta_{\mu\nu}[(\partial_\rho)_b-(\partial_\rho)_a]+\delta_{\mu\rho}[(\partial_\nu)_a-(\partial_\nu)_c]+\delta_{\rho\nu}[(\partial_\mu)_c-(\partial_\mu)_b]\bigg\}
\end{split}
\end{equation}
Then we assume
\begin{align}
\begin{split}
\Gamma_2&=f^{abc}f^{ab'c'}A^b_\mu A^c_\nu A^{b'}_\mu A^{c'}_\nu\\
&=f^{abc}f^{a'b'c}A^a_\mu A^b_\nu A^{a'}_\mu A^{b'}_\nu
\end{split}
\end{align}
And the derivation can be written as
\begin{equation}
\frac{\delta\Gamma_2}{\delta A^a_\mu}=4f^{abc}f^{a'b'c}A^b_\nu A^{a'}_\mu A^{b'}_\nu
\end{equation}
\begin{equation}
\begin{split}
	\frac{\delta\Gamma_2}{\delta A^a_\mu A^b_\nu}&=4f^{abc}f^{a'b'c}A^{a'}_\mu A^{b'}_\nu+4f^{ab''c}f^{bb'c}A^{b''}_\rho \delta_{\mu\nu}A^{b'}_\rho+4f^{ab''c}f^{a'bc}A^{b''}_\nu A^{a'}_\mu\\
&=4f^{abc}f^{a'b'c}A^{a'}_\mu A^{b'}_\nu+4\delta_{\mu\nu}f^{ab'c}f^{bb''c}A^{b'}_\rho A^{b''}_\rho -4f^{ab''c}f^{ba'c}A^{a'}_\mu A^{b''}_\nu
\end{split}
\end{equation}
So now we can obtain the F matrix elements of the glon part
\begin{equation}
\begin{split}
	(F^{AA}_k)^{ab}_{\mu\nu}=&\frac{\delta^2}{A^a_\mu A^b_\nu}(\frac{1}{4}F^a_{\mu\nu}F^a_{\mu\nu} )\\=&Z^{\frac{3}{2}}_{A,k}g_k f^{abc}\bigg\{\delta_{\mu\nu}[(\partial_\rho)_b-(\partial_\rho)_a]A^c_\rho+[(\partial_\nu)_a A^c_\mu-(\partial_\mu)_b A^c_\nu]+(\partial_\mu A^c_\nu-\partial_\nu A^c_\mu)\bigg\}\\&+Z^2_{A,k}
g^2_k [f^{abc}f^{a'b'c}A^{a'}_\mu A^{b'}_\nu+\delta_{\mu\nu}f^{ab'c}f^{bb''c}A^{b'}_\rho A^{b''}_\rho-f^{ab''c}f^{ba'c}A^{a'}_\mu A^{b''}_\nu]
\end{split}
\end{equation}
From the expression of the effective action we find that the interaction between the gluon and meson vanished
\begin{equation}
F^{A\sigma}_k=0
\end{equation}
\begin{equation}
F^{A\pi}_k=0
\end{equation}
Now we derive the gluon-quark vertex. The gluon-quark part of the effective action reads
\begin{equation}
	\Gamma_k\sim Z_{q,k}\bar{q}\gamma_\mu D_\mu q
\end{equation}
The expression of the convariant derivative is
\begin{equation}
	D_\mu=\partial_\mu -iZ^{\frac{1}{2}}_{A,k}g_k A^a_\mu t^a
\end{equation}
So we get
\begin{equation}
\begin{split}
	\Gamma_k&\sim Z_{q,k}\bar{q}\gamma_\mu (-iZ^{\frac{1}{2}}_{A,k}g_k A^a_\mu t^a)q\\
	&=-iZ_{q,k}Z^{\frac{1}{2}}_{A,k}g_k A^a_\mu \bar{q}\gamma_\mu t^a q
\end{split}
\end{equation}
The gluon-quark vertex can be obtained from the action above
\begin{equation}
\begin{split}
	&(F^{Aq}_k)^a_{\mu j}=\frac{\overrightarrow{\delta}}{\delta A^a_\mu}\Gamma_k \frac{\overleftarrow{\delta}}{\delta q_j}=-iZ_{q,k}Z^{\frac{1}{2}}_{A,k}g_k (\bar{q}\gamma_\mu t^a)_j\\
	\Longrightarrow& (F^{Aq}_k)^a_\mu=-iZ_{q,k}Z^{\frac{1}{2}}_{A,k}g_k(\bar{q}\gamma_\mu t^a)
\end{split}
\end{equation}
\begin{equation}
\begin{split}
	&(F^{A\bar{q}}_k)^a_{\mu j}=\frac{\overrightarrow{\delta}}{\delta A^a_\mu}\Gamma_k \frac{\overleftarrow{\delta}}{\delta \bar{q}_j}=iZ_{q,k}Z^{\frac{1}{2}}_{A,k}g_k \,\,_{j}(\gamma_\mu t^a q)\\
	\Longrightarrow&(F^{A\bar{q}}_k)^a_\mu=iZ_{q,k}Z^{\frac{1}{2}}_{A,k}g_k q^T (\gamma_\mu t^a)^T=-(F^{\bar{q}A}_k)^{a\,\,T}_\mu
\end{split}
\end{equation}
\begin{equation}
\begin{split}
	&(F^{qA}_k)^b_{i\nu}=\frac{\overrightarrow{\delta}}{\delta q_i}\Gamma_k\frac{\overleftarrow{\delta}}{\delta A^b_\nu}=iZ_{q,k}Z^{\frac{1}{2}}_{A,k}g_k (\bar{q}\gamma_\nu t^b)_i\\
	\Longrightarrow&(F^{qA}_k)^a_\mu=iZ_{q,k}Z^{\frac{1}{2}}_{A,k}g_k(\gamma_\mu t^a)^T \bar{q}^T=-(F^{Aq}_{k})^{a\,\,T}_\mu
\end{split}
\end{equation}
\begin{equation}
\begin{split}
	&(F^{\bar{q}A}_k)^a_{i\nu}=\frac{\overrightarrow{\delta}}{\delta \bar{q}_i}\Gamma_k \frac{\overleftarrow{\delta}}{\delta A^b_\nu}=iZ_{q,k}Z^{\frac{1}{2}}_{A,k}g_k \,\,_{i}(\gamma_\nu t^b q)\\
	\Longrightarrow&(F^{\bar{q}A}_k)^a_\mu=-iZ_{q,k}Z^{\frac{1}{2}}_{A,k}g_k (\gamma_\mu t^a q)
\end{split}
\end{equation}


\subsection{Meson vertex}
The corresponding effective action of the meson part is
\begin{equation}
	\Gamma_k \sim V_k(\rho)-c\sigma
\end{equation}
the $\rho$ stands for the square of the meson field
\begin{equation}
	\rho=\frac{1}{2}\phi^2_i
\end{equation}
and the definition of the meson field are
\begin{equation}
	\phi_{i=0}=\sigma ,\,\,\,\,\phi_i=\pi_i\,\,\,\,i=1,2,3
\end{equation}
Then we define 
\begin{equation}
\begin{cases}
\phi_i=\bar{\phi}_0+\sigma,\quad for\,\,\,\, i=0\\
\phi_i=\pi_i,\quad for\,\,\,\, i=1,2,3\\
\end{cases}
\end{equation}
The derivation of the effective action 
\begin{equation}
	\frac{\delta \Gamma_k}{\delta \phi_i}=V^{'}_k(\rho)\phi_i-c\delta_{i,0}
\end{equation}
\begin{equation}
	\frac{\delta^2\Gamma_k}{\delta\phi_i\delta\phi_j}=V^{'}_k(\rho)\delta_{i,j}+V^{''}_k(\rho)\phi_i\phi_j
\end{equation}
Now expand the equation above at the point $\phi_i=\bar{\phi}_i$
\begin{equation}
\bar{\phi}_i=
	\begin{cases}
\bar{\phi}_0\,\,\,\,\,\,\,\, i=0\\
0\,\,\,\,\,\,\,\, i=1,2,3\\
\end{cases}
\end{equation}
\begin{equation}
	(\Gamma^{\phi\phi}_k)_{ij}=(\Gamma^{\phi\phi}_k)_{ij}\bigg\arrowvert_{\bar{\phi}}+\frac{\delta^3\Gamma_k}{\delta\phi_i\delta\phi_j\delta\phi_k}\bigg\arrowvert_{\bar{\phi}}(\phi_k-\bar{\phi}_k)+\frac{1}{2}\frac{\delta^4\Gamma_k}{\delta\phi_i\delta\phi_j\delta\phi_k\delta\phi_l}\bigg\arrowvert_{\bar{\phi}}(\phi_k-\bar{\phi}_k)(\phi_l-\bar{\phi}_l)+\dots
\end{equation}
Here we only consider to the two derivative of the effective potential
\begin{equation}
\begin{split}
\frac{\delta^3\Gamma_k}{\delta\phi_i\delta\phi_j\delta\phi_k}&=V''_k(\rho)\phi_k\delta_{ij}+V''_k(\rho)(\delta_{ik}\phi_i+\delta_{jk}\phi_i)+V'''_k(\rho)\phi_i\phi_j\phi_k\\
&=V''_k(\rho)(\delta_{ij}\phi_k+\delta_{ik}\phi_j+\delta_{jk}\phi_i)+V'''_k(\rho)\phi_i\phi_j\phi_k
\end{split}
\end{equation}
\begin{equation}
\begin{split}
\frac{\delta^4\Gamma_k}{\delta\phi_i\delta\phi_j\delta\phi_k\delta\phi_l}&=V''_k(\rho)(\delta_{ij}\delta_{kl}+\delta_{ik}\delta_{jl}+\delta_{jk}\phi_{il})\\
&+V'''_k(\rho)(\delta_{ij}\phi_k\phi_l+\delta_{ik}\phi_j\phi_l+\delta_{jk}\phi_i\phi_l)\\
&+V'''_k(\rho)(\delta_{il}\phi_j\phi_k+\delta_{jl}\phi_i\phi_k+\delta_{kl}\phi_i\phi_j)\\
&+V''''_k(\rho)\phi_i\phi_j\phi_k\phi_l\\
&=V''_k(\rho)(\delta_{ij}\delta_{kl}+\delta_{ik}\delta_{jl}+\delta_{jk}\phi_{il})\\
&+V^{(3)}_{k}(\rho)(\delta_{ij}\phi_k\phi_l+\delta_{ik}\phi_j\phi_l+\delta_{jk}\phi_i\phi_l+\delta_{il}\phi_j\phi_k+\delta_{jl}\phi_i\phi_k+\delta_{kl}\phi_i\phi_j)\\
&+V^{(4)}_{k}(\rho)\phi_i\phi_j\phi_k\phi_l
\end{split}
\end{equation}
The definition of the Sigma meson mass is
\begin{equation}
\begin{split}
m^2_{\sigma}=(\Gamma^{\sigma\sigma}_k)\big|_{\bar{\phi}}=(\Gamma^{\phi\phi}_k)_{00}\big|_{\bar{\phi}}&=V'_k(\rho)+V''_k(\rho)\bar{\phi}_0\bar{\phi}_0\\
&=V'_k(\rho)+2\rho V''_k(rho)
\end{split}
\end{equation}
And the Pion mass is
\begin{equation}
m^2_{\pi}=(\Gamma^{\pi\pi}_k)_{ij}\big|_{\bar{\phi}}=(\Gamma^{\phi\phi}_k)_{ij}\big|_{\bar{\phi}}=V'_k(\rho)\delta_{ij}
\end{equation}
Then the other derivation part of the Pion and Sigma meson can be written as
\begin{equation}
\begin{split}
(\Gamma^{\sigma\pi}_k)\big|_{\bar{\phi}}&=0\\
\frac{\delta^3\Gamma_k}{\delta\sigma\delta\sigma\delta\sigma}\bigg|_{\bar{\phi}}&=V''_k(\rho)(3\bar{\phi}_0)+V^{(3)}_{k}(\rho)\bar{\phi}_0^3\\
&=3(2\rho)^{\frac{1}{2}}V''_k(\rho)+(2\rho)^{\frac{3}{2}}V^{(3)}_{k}(\rho)\\
\frac{\delta^3\Gamma_k}{\delta\sigma\delta\sigma\delta\pi_k}\bigg|_{\bar{\phi}}&=0\\
\frac{\delta^3\Gamma_k}{\delta\sigma\delta\pi_j\delta\sigma}\bigg|_{\bar{\phi}}&=0\\
\frac{\delta^3\Gamma_k}{\delta\sigma\delta\pi_j\delta\pi_k}\bigg|_{\bar{\phi}}&=V''_k(\rho)\delta_{jk}\bar{\phi}_0=\delta_{jk}(2\rho)^{\frac{1}{2}}V''_k(\rho)\\
\frac{\delta^3\Gamma_k}{\delta\pi_i\delta\pi_j\delta\sigma}\bigg|_{\bar{\phi}}&=\delta_{ij}(2\rho)^{\frac{1}{2}}V''_k(\rho)\\
\frac{\delta^3\Gamma_k}{\delta\pi_i\delta\pi_j\delta\pi_k}\bigg|_{\bar{\phi}}&=0
\end{split}
\end{equation}
If we only take the lowest derivation into account

%%%%%%%%%%%%%%%%%%%%%%%%%%%%%%%%%%%%%
%附录 n点函数的傅立叶变换
\begin{appendices}
\section{Fourier transform of n-point function}
Consider a general n-point function $V(x_1,x_2,...,x_n)$,
Its Fourier transformation can be written as:
\begin{equation}
V(x_1,x_2,...,x_n)=\int\cfrac{d^4p_1}{(2\pi)^4}
\cfrac{d^4p_2}{(2\pi)^4}...\cfrac{d^4p_n}{(2\pi)^4}
V(p_1,p_2,...p_n)e^{i(p_1x_1+p_2x_2+...p_nx_n)}
\end{equation}
if $V$ satisfies the following properties:
\begin{equation}
V(x_1,x_2,...x_n)=V(x_1-x_n,x_2-x_n,...,0)
\end{equation}
The value of $V$ depends only on the relative value of the coordinates, 
so we can obtain
\begin{equation}
\begin{split}
V(p_1,p_2,...p_n)=&\int d^4x_1d^4x_2...d^4x_nV(x_1,x_2,...x_n)
e^{-i(p_1x_1+p_2x_2+...p_nx_n)}\\
=&\int d^4x_1d^4x_2...d^4x_nV(x_1-x_n,x_2-x_n,...0)
e^{-i(p_1x_1+p_2x_2+...p_nx_n)}\\
=&\int d^4x_1d^4x_2...d^4x_nV(x_1,x_2,...0)
e^{-i[p_1(x_1+x_n)p_2(x_2+x_n)+...p_{n-1}(x_{n-1}+x_n)]}\\
=&\int d^4x_1d^4x_2...d^4x_nV(x_1,x_2,...0)
e^{-i[p_1x_1+p_2x_2+...p_{n-1}x_{n-1}+(p_1+p_2+...+p_n)x_n]}\\
=&(2\pi)^4\delta^4(p_1+p_2+...p_n)\int d^4x_1d^4x_2...d^4x_{n-1}
V(x_1,x_2,...,0)e^{-i(p_1x_1+p_2x_2+...p_{n-1}x_{n-1})}
\end{split}
\end{equation}
So we can redefine $V(p_1,p_2,...p_n)$ as
\begin{equation}
V(p_1,p_2,...p_n)=(2\pi)^4\delta^4(p_1+p_2+...p_n)
V(p_1,p_2,...p_{n-1},-(p_1+p_2+...p_{n-1}))
\end{equation}
\begin{equation}
\begin{split}
V(x_1,x_2,...x_n)
&=\int\cfrac{d^4p_1}{(2\pi)^4}\cfrac{d^4p_2}{(2\pi)^4}...
\cfrac{d^4p_n}{(2\pi)^4}(2\pi)^4\delta^4(p_1+p_2...+p_n)
V(p_1,p_2,...-(p_1,p_2+...p_{n-1}))e^{i(p_1x_1+p_2x_2+...p_nx_n)}\\
&=\int\cfrac{d^4p_1}{(2\pi)^4}\cfrac{d^4p_2}{(2\pi)^4}...
\cfrac{d^4p_{n-1}}{(2\pi)^4}V(p_1,p_2,...-(p_1,p_2+...p_{n-1}))
e^{i[p_1(x_1-x_n)+p_2(x_2-x_n)+...p_{n-1}(x_{n-1}-x_n)]}
\end{split}
\end{equation}
\end{appendices}
%%%%%%%%%%%%%%%%%%%%%%%%%%%%%%%%%%%%%%%%
%%%%%%%%%%%%%%%%%%%%%%%%%%%%%%%%%%%%%%%%
\end{document}
