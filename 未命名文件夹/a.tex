\documentclass[UTF8]{ctexart}
\usepackage{multicol,ifthen,booktabs,amsmath,amsfonts,bm,mathrsfs,amssymb}
\usepackage{times,mathptmx}
\usepackage{geometry}
\renewcommand\baselinestretch{1.5}\protect
\abovedisplayshortskip 3 pt
\belowdisplayshortskip 3 pt
\geometry{left=2cm,right=2cm,top=3cm,bottom=3cm}
\begin{document}
\section{Wetterich Equation}

考虑红外截断的生成泛函
\begin{equation}\label{eq0001}
Z_k[J]=\exp(W_k[J])=\int [d\phi]\exp\{-S[\phi]-\Delta S_k[\phi]]+J^a\phi_a\}
\end{equation}
这里,
我们用$\phi$表示所有各种类型的场。
抽象指标$a$代表各种自由度,
包括不同的场,
同种场的不同分量,
以及其它各种分离和连续的自由度,
比如时空坐标或动量指标。

$S[\phi]$是经典的作用量,
$J^a$是$\phi_a$的源,
$\Delta S_k[\phi]$是红外截断,
它的作用是截断$p^2<k^2$的量子涨落。
通常我们选取二次项的形式(类似于质量项)
来实现这种红外截断
\begin{equation}
\Delta S_k[\phi]=\cfrac{1}{2}\phi_aR_k^{ab}\phi_b
\end{equation}
其中$R_k^{ab}=R_k^{ba}$($a,b$ 是玻色场指标),
$R_k^{ab}=-R_k^{ba}$($b,a$ 是费米场指标),
我们这里以单个标量场为例,
在坐标空间中
\begin{equation}
\Delta S_k[\phi]=
\cfrac{1}{2}\int d^4xd^4y\varphi(x)R_k(x,y)\varphi(y)
\end{equation}
根据附录1.A,
得
\begin{equation}
\begin{split}
R_k(x,y)= & \int\cfrac{d^4p}{(2\pi)^4}R_k(p,-p)e^{ip(x-y)}\\
= & \int\cfrac{d^4p}{(2\pi)^4} R_k(p)e^{ip(x-y)}
\end{split}
\end{equation}
\begin{equation}
\varphi(x)=\int\cfrac{d^4p}{(2\pi)^4}e^{ipx}\varphi(p)
\end{equation}
\begin{equation}
\begin{split}
\Delta S_k[\varphi]=&\cfrac{1}{2}\int d^4xd^4y\varphi(x)R_k(x,y)\varphi(y)\\
=&\cfrac{1}{2}\int d^4xd^4y
\int\cfrac{d^4p_1}{(2\pi)^4}e^{ip_1x}\varphi(p_1)
\int\cfrac{d^4p}{(2\pi)^4}R_k(p)e^{ip(x-y)}
\int\cfrac{d^4p_2}{(2\pi)^4}e^{ip_2y}\varphi(p_2)\\
=&\cfrac{1}{2}\int\cfrac{d^4p}{(2\pi)^4}\varphi(-p)R_k(p)\varphi(p)\\
=&\cfrac{1}{2}\int\cfrac{d^4q}{(2\pi)^4}\varphi(-q)R_k(q)\varphi(q)
\end{split}
\end{equation}
对于确定的$q$,
regulator $R_k(q)$满足下面的性质
\begin{equation}
R_{k\rightarrow\infty}(q)\rightarrow\infty
\quad
R_{k\rightarrow 0}(q)\rightarrow 0
\end{equation}
为了压低$q^2<k^2$动量模式的涨落,
而让高动量模式的涨落没有影响,
我们可以选择
\begin{equation}
R_k(q)|_{q^2<k^2}\sim k^2,
\quad
R_k(q)|_{q^2>k^2}\sim 0
\end{equation}
比如
\begin{equation}
R_k(q)\sim\cfrac{q^2}{e^{\frac{q^2}{k^2}}-1}
\end{equation}
当然,
我们可以选择其他形式的regulator,
从式\eqref{eq0001}可以得到
\begin{equation}
\begin{split}
\cfrac{\delta W_k[J]}{\delta J^a}=&\cfrac{1}{Z_k}
\cfrac{\delta Z_k[J]}{\delta J^a}\\
=&\cfrac{1}{Z_k}\int [d\phi]\phi_a\exp\{-S[\phi]
-\Delta S_k[\phi]+J^a\phi_a\}\\
=&\langle\phi_a\rangle
\end{split}
\end{equation}
在下面的讨论中,
在不引起混淆的情况下,
我们仍然用$\phi_a$表示$\langle\phi_a\rangle$。
同样
\begin{equation}
\cfrac{\delta^2W_k[J]}{\delta J^b\delta J^a}
=\langle \phi_b\phi_a\rangle_c\equiv G_{ba}^k
\end{equation}
下标c代表连通图,
G是标度k依赖的传播子。

下面,
我们对连通图的生成泛函$W_k[J]$作Legendre变换。
得到单粒子不可约的(1PI)图的生成泛函,
即有效作用量
\begin{equation}\label{eq0002}
\Gamma_k[\phi]=-W_k[J]+J^a\phi_a-\Delta S_k[\phi]
\end{equation}
注意,
这里的$\phi_a\equiv\langle\phi_a\rangle$

为了将玻色场和费米场一并考虑,
我们引入下面的记号
\begin{equation}
J^a\phi_a=r_b^a\phi_aJ^b
\end{equation}
其中
\begin{equation}
r_b^a=(-1)^{ab}\delta_b^a
\end{equation}
\begin{equation}
(-1)^{ab}\equiv
\begin{cases}
-1,\quad for \quad a,b \quad fermion\\
1,\quad for \quad a,b \quad beson\\
\end{cases}
\end{equation}
这样,
从式\eqref{eq0002}得到
\begin{equation}
\cfrac{\delta(\Gamma_k[\phi]+\Delta S_k[\phi])}{\delta\phi_a}
=r_b^aJ^b
\end{equation}
对上式两边再求$J$的微商,
得到
\begin{equation}
\cfrac{\delta^2(\Gamma_k[\phi]+\Delta S_k[\phi])}
{\delta J^b\delta\phi_a}
=r_b^a
\end{equation}

\subsection*{附录1.A~~~ n点函数的Fourier变换}
考虑一个一般的n点函数$V(x_1,x_2,...,x_n)$,
它的Fourier变换为
\begin{equation}
V(x_1,x_2,...,x_n)=\int\cfrac{d^4p_1}{(2\pi)^4}
\cfrac{d^4p_2}{(2\pi)^4}...\cfrac{d^4p_n}{(2\pi)^4}
V(p_1,p_2,...p_n)e^{i(p_1x_1+p_2x_2+...p_nx_n)}
\end{equation}
假如V满足下面的性质
\begin{equation}
V(x_1,x_2,...x_n)=V(x_1-x_n,x_2-x_n,...,0)
\end{equation}
即V的函数值只依赖于坐标的相对值,
那么
\begin{equation}
\begin{split}
V(p_1,p_2,...p_n)=&\int d^4x_1d^4x_2...d^4x_nV(x_1,x_2,...x_n)
e^{-i(p_1x_1+p_2x_2+...p_nx_n)}\\
=&\int d^4x_1d^4x_2...d^4x_nV(x_1-x_n,x_2-x_n,...0)
e^{-i(p_1x_1+p_2x_2+...p_nx_n)}\\
=&\int d^4x_1d^4x_2...d^4x_nV(x_1,x_2,...0)
e^{-i[p_1(x_1+x_n)p_2(x_2+x_n)+...p_{n-1}(x_{n-1}+x_n)]}\\
=&\int d^4x_1d^4x_2...d^4x_nV(x_1,x_2,...0)
e^{-i[p_1x_1+p_2x_2+...p_{n-1}x_{n-1}+(p_1+p_2+...+p_n)x_n]}\\
=&(2\pi)^4\delta^4(p_1+p_2+...p_n)\int d^4x_1d^4x_2...d^4x_{n-1}
V(x_1,x_2,...,0)e^{-i(p_1x_1+p_2x_2+...p_{n-1}x_{n-1})}
\end{split}
\end{equation}
这样我们就可以将$V(p_1,p_2,...p_n)$重新定义为
\begin{equation}
V(p_1,p_2,...p_n)=(2\pi)^4\delta^4(p_1+p_2+...p_n)
V(p_1,p_2,...p_{n-1},-(p_1+p_2+...p_{n-1}))
\end{equation}
\begin{equation}
\begin{split}
V(x_1,x_2,...x_n)
&=\int\cfrac{d^4p_1}{(2\pi)^4}\cfrac{d^4p_2}{(2\pi)^4}...
\cfrac{d^4p_n}{(2\pi)^4}(2\pi)^4\delta^4(p_1+p_2...+p_n)
V(p_1,p_2,...-(p_1,p_2+...p_{n-1}))e^{i(p_1x_1+p_2x_2+...p_nx_n)}\\
&=\int\cfrac{d^4p_1}{(2\pi)^4}\cfrac{d^4p_2}{(2\pi)^4}...
\cfrac{d^4p_{n-1}}{(2\pi)^4}V(p_1,p_2,...-(p_1,p_2+...p_{n-1}))
e^{i[p_1(x_1-x_n)+p_2(x_2-x_n)+...p_{n-1}(x_{n-1}-x_n)]}
\end{split}
\end{equation}
\end{document}
